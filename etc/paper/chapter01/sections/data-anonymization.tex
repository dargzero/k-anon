In the previous chapters we briefly introduced the sociological and political aspects of privacy.
Now we will proceed to discuss the concept of \textbf{\textit{anonymization}} and its technical aspects.
So how exactly can we define anonymization?

\paragraph{Definition} \textsc{Data anonymization} has been defined as a process by which personal data is irreversibly altered in such a way that a data subject can no longer be identified directly or indirectly, either by the data controller alone or in collaboration with any other party~\cite{wiki04, iso01}.

A typical \textit{data controller} can be a hospital storing medical data about its patients.
The data is then released to third parties or government entities, for example to draw conclusions in an epidemic outbreak.
We can see, that some parts of the data like address, age, race, gender --- although personal --- can be necessary to draw accurate statistical conclusions.
To allow the extraction of useful statistics while also protecting the privacy of its patients, the hospital will need to alter the data before handing it over.

