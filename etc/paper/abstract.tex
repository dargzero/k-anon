\section*{Abstract}\label{sec:abstract-en}

Collected user and customer information is considered sensitive data. In our modern world several companies offer easy to use, convenient services in exchange for our personal data and user habits. Collection of personal data has seen a tremendous growth, and collected personal data is now a commodity.

The European Union is attempting to enforce strict regulation regarding personal data. One example for this is the recently adopted regulation called GDPR, which lays down the foundation for handling of personal data, and controls how such data can be stored and disclosed. It demands, that data disclosed to a third party must always be anonymized, which places a serious demand on companies, especially larger ones.

k-anonymity is a theoretical model, which describes when a data set can be considered anonymous. Professional literature knows several implementations for the model, one of which is the algorithm published by authors Aggarwal, Féder, Kenthapadi, Motwani, Panigrahy, Thomas, Zhu in 2005. The essence of this algorithm is that data is represented as a graph, then it is partitioned based on a cost-function derived from generalization.

This work introduces the problem of data anonymization, and a practical implementation of the graph based algorithm in the Go programming language. It gives a detailed description of the algorithm, applied technical and implementation details, and runtime performance. Finally it showcases several examples of using the product, including showcasing integration into a third party system.

\clearpage
\thispagestyle{empty}
\pagebreak

\section*{Összefoglaló}\label{sec:abstract-hu}

A felhasználói és ügyfél adatok személyiségi jogok tekintetében érzékeny adatok. Mai modern világunkban számos cég kínál kényelmesen használható, ingyenes szolgáltatásokat, cserébe a személyes adatainkért és felhasználói szokásainkért. A személyes adatok gyűjtése rendkívüli mértékben felgyorsult és a begyűjtött adatok portékává váltak.

Az Európai Unió egyre szigorúbb szabályozással igyekszik fellépni ezen a téren. Ennek egy példája a nemrég hatályba lépett GDPR szabályozás, mely lefekteti a személyes adatok kezelésének keretrendszerét, valamint szabályozza az ilyen jellegű adatok tárolásának és elérhetővé tételének követelményeit. Megköveteli ugyanis minden harmadik fél számára elérhetővé tett személyes adat anonimizálását, mely komoly erőforrásokat igényel minden cégtől, különösképpen a nagyvállalatoktól.

A k-anonimitás egy elméleti modell, amely leírja, mikor tekinthető egy adathalmaz anonimizáltnak. A modell implementálására számos megoldást ismer az irodalom. Ezek egyike az Aggarwal, Féder, Kenthapadi, Motwani, Panigrahy, Thomas, Zhu szerzők által 2005-ben publikált algoritmus, amely lényege az adatok gráfként történő reprezentálása, majd a gráf particionálása egy, az adat generalizálásából származtatott él-költség függvény alapján.

Ez a munka ismerteti az anonimizálás problémakörét, valamint ezen gráf algoritmus egy gyakorlatban használható implementációját Go programozási nyelven. Részletesen kitér az algoritmus műküdésére, az alkalmazott technikai megoldásokara és a program teljesítményére. Végül számos konkrét példát mutat a termék alkalmazására, beleértve egy külső rendszerbe történő integráció bemutatását.

\clearpage
\thispagestyle{empty}