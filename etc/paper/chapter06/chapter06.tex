\section{Improved algorithm}

The original work on which our graph algorithm implementation is based on presents an improved algorithm for the \(k=2\) and \(k=3\) special cases~\cite{aggarwal}. If the use case often requires the use of \(k=2\) or \(k=3\) anonymity, it might be beneficial to extend the anonymizer to handle these special cases differently.

\subsection{Improved algorithm for k=2}\label{subsec:improved_algorithm}
The method presented in Section~\ref{ch:chapter_algorithm} gives a \textit{3-approximation} algorithm for the \(k=2\) case. However, there is polynomial time \textit{1.5-approximation} algorithm when \(k=2\) and using a \textit{binary alphabet}. Note, that for a binary alphabet generalization is equivalent to suppression~\cite{aggarwal}.

This special algorithm creates the cost-graph by considering the \textit{Hamming distance} between the vectors represented by two nodes in the graph. Then we obtain a \textbf{minimum-weight} \([1,2]\)-factor \textit{F} of \textit{G}.

\paragraph{Definition} A \([1,2]\)-factor of an edge-weighted graph \textit{G} is defined to be a spanning sub-graph \textit{F} of \textit{G} such that each vertex in \textit{F} has a degree of 1 or 2~\cite{aggarwal}.

It has been shown, that finding a minimum-weight \([1,2]\)-factor of a graph can be computed in polynomial time (Cornuejols, 1988).

We treat each component in the obtained \textit{F} graph as a \textit{cluster}, and for each vector in the data retain bits where they are in the same cluster, and suppress bits where they aren't.

\paragraph{Theorem} the number of \texttt{*}s introduced by the above algorithm is at most \(1.5\) times the number of \texttt{*}s in an \textit{optimal} 2-anonymity solution~\cite{aggarwal}.

\subsection{Improved algorithm for k=3}

Also for a binary alphabet, it is possible to give a \textit{2-approximation} algorithm for the \(k=3\) case~\cite{aggarwal}. This algorithm uses a similar algorithm to~\ref{subsec:improved_algorithm}, with the exception of now calculating \textbf{minimum-weight} 2-factor \textit{F} spanning sub-graph, which can also be done in polynomial time.

\paragraph{Theorem} the resulting 2-factor \textit{F} can be transformed into a 2-approximation solution for 3-anonymity~\cite{aggarwal}.


\section{Performance}

The result of the performance benchmarks in~\ref{sec:benchmarks} shows, that the \textit{FloatRangeGeneralizer} and \textit{HierarchyGeneralizer} in particular suffer from poor performance, especially when the respective range size and node count is larger. To a smaller extent, the same problem can be observed with the \textit{PrefixGeneralizer} when the word count is more than 5000 words.

Improving the performance of these generalizers might be possible by reducing the step count needed to locate an item among the possible list of partitions on a given level. It might also be possible to introduce a better data structure for the generalizers that can eliminate the need to scan through all partitions on each level, thus reducing search step count significantly for larger input.